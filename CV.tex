% Created 2024-01-24 mié 14:44
% Intended LaTeX compiler: pdflatex
\documentclass[11pt]{article}
\usepackage[rm]{roboto}
\usepackage[defaultsans]{lato}
\usepackage{paracol}
altacv
\usepackage[spanish, activeacute, american]{babel}
\columnratio{0.6} % Set the left/right column width ratio to 6:4.
\author{Carlos Eduardo Martínez Auilar}
\date{\today}
\title{Curriculum Vitae}
\hypersetup{
 pdfauthor={Carlos Eduardo Martínez Auilar},
 pdftitle={Curriculum Vitae},
 pdfkeywords={},
 pdfsubject={},
 pdfcreator={Emacs 29.1 (Org mode 9.7)}, 
 pdflang={English}}
\begin{document}

\name{Carlos Eduardo Martínez Aguilar}
\photoR{2.8cm}{photo_2020-02-05_23-29-44.jpg}
\tagline{PhD Researcher}

\personalinfo{
    %\homepage{www.aidanscannell.com}
    \email{cmartineza@ciencias.unam.mx}
    \phone{+52 55272020857}
    \location{CDMX, MX}
    \github{buddharta}
    %\linkedin{aidan-scannell-82522789/}
}
\makecvheader

\section{Información General}
\label{sec:orgaf62938}
\begin{itemize}
\item Nombre completo: Carlos Eduardo Martínez Aguilar
\item Nacionalidad: Mexicana
\item Lugar de nacimiento: Ciudad de México
\item Fecha de nacimiento: 14 de julio de 1992
\end{itemize}
\section{Idiomas}
\label{sec:org3d75503}
\begin{itemize}
\item Español, nativo
\item Inglés, fluido (Puntuación TOEFL iBT: 96)
\item Francés, intermedio
\item Japonés, básico
\end{itemize}

\section{Educación}
\label{sec:orgf4c6971}
\begin{itemize}
\item 2011-2016. Licenciatura en Matemáticas (con mención honorífica). Facultad de Ciencias, Universidad Nacional Autónoma de México (UNAM), Ciudad de México.
\begin{itemize}
\item Título de tesis: Restricciones universales para grupos fuchsianos.
\item Asesor: Antonio Lascurain Orive.
\end{itemize}

\item 2016-2019. Maestría en Ciencias Matemáticas. Instituto de Matemáticas, Universidad Nacional Autónoma de México (UNAM), Ciudad de México.
\begin{itemize}
\item Título de tesis: Estructuras proyectivas con holonomía fuchsiana.
\item Asesores: Adolfo Guillot Santiago y Antonio Lascurain Orive.
\end{itemize}

\item 2020-presente. Doctorado en Ciencias Matemáticas. Instituto de Matemáticas, Universidad Nacional Autónoma de México (UNAM), Ciudad de México.
\begin{itemize}
\item Título de tesis: Un estudio de Foliaciones Holomorfas en Variedades Kahler compactas.
\item Asesores: Alberto Verjovsky Santiago.
\end{itemize}
\end{itemize}

\section{Premios}
\label{sec:orgab8a99c}
\begin{itemize}
\item 2017. Medalla Gabino Barreda, Universidad Nacional Autónoma de México (UNAM), Ciudad de México, México.
\begin{itemize}
\item Otorgada por ser el mejor de la generación 2012-2016 en matemáticas (obteniendo la calificación más alta de la generación GPA: 10/10).
\end{itemize}
\end{itemize}

\section{Talleres, Escuelas y Seminarios}
\label{sec:orgbe9b13f}
\begin{itemize}
\item Dic 2014. Escuela de Nudos y 3-variedades. Guanajuato, Guanajuato. Centro de Investigación en Matemáticas, A.C.
\item May 2015. Encuentro de Geometría Algebraica. Guanajuato, Guanajuato. Centro de Investigación en Matemáticas, A.C.
\item Jul 2016. XIII Escuela de Verano en Matemáticas. Unidad Cuernavaca del Instituto de Matemáticas, UNAM.
\item Ago 2016. Taller sobre grupos kleinianos y temas relacionados. Unidad Cuernavaca del Instituto de Matemáticas, UNAM.
\end{itemize}

\section{Becas}
\label{sec:org40fd066}
\begin{itemize}
\item 2016-2018: Beca Nacional para Maestría en Ciencias Matemáticas - Consejo Nacional de Ciencia y Tecnología (CONACyT).

\item 2020-2024: Beca Nacional para Doctorado en Ciencias Matemáticas - Consejo Nacional de Ciencia y Tecnología (CONACyT).
\end{itemize}
\section{Artículos}
\label{sec:org7c0ea5c}
\begin{itemize}
\item Errett Bishop theorems on Complex Analytic Sets: Chow's Theorem Revisited, and Foliations of Compact Leaves on Kähler Manifolds
\end{itemize}
\section{Empleo}
\label{sec:org9c4ea42}
\begin{itemize}
\item Asistente de enseñanza para estudiantes, Unidad de Posgrado Universidad Nacional Autónoma de México, asistente de profesor de Análisis Funcional.
\end{itemize}

\begin{enumerate}
\item Experiencia
\label{sec:org9e4015f}
\begin{itemize}
\item Enseñanza como asistente de profesor en la Facultad de Ciencias, Universidad Nacional Autónoma de México (FC-UNAM), de los cursos:
\begin{itemize}
\item Álgebra Superior I (Álgebra Básica).
\item Álgebra Superior II.
\item Variable Compleja I (Análisis Complejo).
\item Ecuaciones Diferenciales Ordinarias I (Ecuaciones Diferenciales Ordinarias).
\end{itemize}
\end{itemize}
\end{enumerate}
\end{document}